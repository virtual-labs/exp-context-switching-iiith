% Created 2024-02-09 Fri 13:44
% Intended LaTeX compiler: pdflatex
\documentclass[11pt]{article}
\usepackage[utf8]{inputenc}
\usepackage[T1]{fontenc}
\usepackage{graphicx}
\usepackage{grffile}
\usepackage{longtable}
\usepackage{wrapfig}
\usepackage{rotating}
\usepackage[normalem]{ulem}
\usepackage{amsmath}
\usepackage{textcomp}
\usepackage{amssymb}
\usepackage{capt-of}
\usepackage{hyperref}
\author{Bhargavi}
\date{\today}
\title{Conceptual model for Scheduling}
\hypersetup{
 pdfauthor={Bhargavi},
 pdftitle={Conceptual model for Scheduling},
 pdfkeywords={},
 pdfsubject={},
 pdfcreator={Emacs 27.1 (Org mode 9.3)}, 
 pdflang={English}}
\begin{document}

\maketitle
\tableofcontents


\section{Aim of the experiment}
\label{sec:org670a172}
This experiment primarily aims at bringing an in depth understanding of the context switching mechanism in the OS and to introduce the student to the ideology of viewing a system as a transitional state model with simply changes depending on the action taken by the user or any other internal actions.

\section{What will the student do in this experiment?}
\label{sec:org250bee3}
\begin{itemize}
\item The student is going to play different roles at different times during the whole simulation. The roles played by the student are mainly 'User', 'Hardware', and 'Kernel'.
\item Aided with the prompts and feedbacks provided as a response to their previous actions, the student needs to perform right actions to succesfully do context switch between two process.
\end{itemize}

\section{Constraints/Barriers}
\label{sec:orgc625f6d}
\begin{itemize}
\item The experiment doesn't provide a diverse options of interrupts and syscalls.
\item We have only two fixed processes to observe the context switching mechanism and the student is not given a choice of adding their own programs.
\end{itemize}

\section{Outcomes of the experiment}
\label{sec:org4a967ba}
\begin{itemize}
\item Student will be able to appreciate how different actions perfromed by different modes and subsytems operate together to complete the task of context switching.
\item To develope an high level abstract overview of the actors interacting in the context switching mechanism.
\end{itemize}
\end{document}
